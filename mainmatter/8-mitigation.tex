\section{Meltdown mitigations}
Since Meltdown is a micro-architectural vulnerability, there is no software update that can completaly make mahcines secured from Meldown attacks.
Not even KAISER (also known as PTI, page-table isolation, or KPTI on Linux kernel) which is the proposed
mitigation by meltdown researchers make machines secure. That's because
Mletdown, and Spectre, acts on hardware and bypasses the hardware-enforced isolation of security domains.
A countermeasure would be to completely disable out-of-order execution but this will make processors slow enough to make any modern CPU parallelism mechanism
completely useless and the performance impact would be devastating. As of 2022, PTI (KAISER) is enabled by default on Linux kernels as a countermeasure to Meltdown.
\subsection{KAISER}
KAISER (Kernel Address Isolation to have Side-channels Efficiently Removed) was not originally intended for Meltdown, but has as side effect the mitigation of it
since KAISER prevents side channel attacks breaking KASLR. But this has its own limitations: first of all, performances will decrease since every context switch will
need more clock cycles for address mappings; second, there is still a residual attack surface for Meltdown since several privileged memory locations are required
to be mapped in user space. However, these memory locations do not contain any secrets, but they might contain pointers to Kernel Address space. This information, if leaked,
is enough to break KASLR, as the randomization can be calculated from the pointer value.
