\section{Address Spaces}
Usually, CPUs support virtual address spaces to isolate processes from each other and to let
compilers use logical addresses instead of directly accessing physical memory addresses. Virtual addresses
are then translated to physical addresses. For optimization of memory usage, paging is also used
to reduce memory usage. Paging is also used to separate User Space addresses from Kernel Mode addresses,
in order to let only privileged processes to access kernel address space. Translation tables are used in order to define virtual to phisical mappings
and also protection properties such as readable, writeble, executable and whether the page is accessible by
user or not (meaning that only kernel mode processes can access the page).
Every proces has its own translation table which is held on a special CPU register, so "on each context switch the \textbf{operating system} updates
this register with the next process' translation table address in order to implement per process virtual address spaces".
Each virtual address space itself is split into a user and a kernel part.
\subsubsection{Exploitation and mitigation}
Attacks that are targeting memory corruption bugs often requires the knowledge of addresses of specific data.
ASLR mitigation has been introduced to randomize address space layout in order to obfuscate memory mapping to
attackers. KASLR (Kernel Address Layout Randomization) was introduced to protect the kernel, randomizing the offsets where
drivers are located on every boot, making attacks harder as they now require to guess the location of kernel data structures.
\subsubsection{Side channel attacks}
From Wikipedia, here's a definition of side-channel attack
\begin{quotation}
    In computer security, a side-channel attack is any attack based on extra
    information that can be gathered because of the fundamental way a computer
    protocol or algorithm is implemented, rather than flaws in the design of the
    protocol or algorithm itself or minor, but potentially devastating, mistakes
    or oversights in the implementation.
\end{quotation}
Side-channel attacks allow to detect the exact location of kernel data structures or derandomize ASLR. A combination of
software bug and the knowledge of these addresses can lead to privileged code execution.
More in depth on side channels, there are many ways we can gather information, for example: timing, RF, electromagnetic emissions, and others.
[reference: https://www.youtube.com/watch?v=D1DNz5sNDgE]
Moreover, there's also the Covert Channel attacks, which are a special use case of side channels, where basically we intentionally send information
to a system in order to induce the side effects we want to measure.
[reference: meltdown, https://en.wikipedia.org/wiki/Covert\_channel]
Specifically for our use case, side channels includes: Evict+Time, Prime+Probe and Flush+Reload. We will discuss only the latter.
These attacks are specifically designed to leak information from the cache exploiting timing differences induced by them selfs.
\subsubsection{Flush+Reload}
An attacker frequently flushes a targeted memory locatiojn using the clflush instruction. By measuring the time it takes to Reload
the data, the attacker determines wheteher data was loaded into the cache by another process in the meantime.



